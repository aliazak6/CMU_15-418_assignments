%% You can use this file to create your answer for Exercise 1.  
%% Fill in the places labeled by comments.
%% Generate a PDF document by with the command `pdflatex ex1'.

\documentclass[11pt]{article}


\usepackage{times}
\usepackage{listings}
\usepackage{enumerate}
\usepackage{courier}
\usepackage{hyperref}
\usepackage{xcolor}
\usepackage{graphicx}


%% Values that are specific to a particular term
\newcommand{\thisterm}{Spring 2021}

\newcommand{\dateassigned}{Mon.,~Apr.~12}

%% Printed form of home page that students should use
\newcommand{\visiblecoursehome}{http://www.cs.cmu.edu/\textasciitilde{}418}

%% Link to home page that will stay valid
\newcommand{\actualcoursehome}{http://www.cs.cmu.edu/afs/cs.cmu.edu/academic/class/15418-s21/www}

\newcommand{\datedue}{Mon.,~Apr.~19}





%% Page layout
\oddsidemargin 0pt
\evensidemargin 0pt
\textheight 600pt
\textwidth 469pt
\setlength{\parindent}{0em}
\setlength{\parskip}{1ex}

%% Colored hyperlink 
\newcommand{\cref}[2]{\href{#1}{\color{blue}#2}}

%% Customization to listing
\lstset{basicstyle=\ttfamily\small,language=C,morekeywords={cilk_synch,cilk_spawn}}

%% Enumerate environment with alphabetic labels
\newenvironment{choice}{\begin{enumerate}[A.]}{\end{enumerate}}
%% Environment for supplying answers to problem
\newenvironment{answer}{\begin{minipage}[c][1.5in]{\textwidth}}{\end{minipage}}


\begin{document}
                          
\vspace*{0.3in}                            
\begin{center}
\LARGE
15-418/618 \thisterm{} \\
Exercise 8
\end{center}

\begin{center}
\Large        
\begin{tabular}{ll}
\hline             
Assigned: & \dateassigned{}  \\
Due: &  \datedue{}, 11:00~pm  \\
\hline       
\end{tabular}
\end{center} 

\section*{Overview}

This exercise is designed to help you better understand the lecture
material and be prepared for the style of questions you will get on
the exams.  The questions are designed to have simple answers.  Any
explanation you provide can be brief---at most 3 sentences.  You
should work on this on your own, since that's how things will be when
you take an exam.

You will submit an electronic version of this assignment to Canvas 
as a PDF file.  For those of you familiar with the \LaTeX{} text 
formatter, you can download the template and configuation files at: 
\begin{center} 
  \cref{\actualcoursehome/exercises/ex7.tex}{\visiblecoursehome/exercises/ex7.tex}\\
  \cref{\actualcoursehome/exercises/config-ex7.tex}{\visiblecoursehome/exercises/config-ex7.tex}
\end{center} 
Instructions for how to use this template are included as comments in 
the file.  Otherwise, you can use this PDF document as your starting 
point.  You can either: 1) electronically modify the PDF, or 2) print 
it out, write your answers by hand, and scan it.  In any case, we 
expect your solution to follow the formatting of this document. 

\newpage 

\section*{Problem 1: Heterogeneous Parallelism}

\begin{choice}
\item Recall the plots shown on lecture 21 slide 7. Qualitatively describe how the plots would change if the perf(r) function were changed to perf($r$) = $r$ and perf($r$) = $\log r$ instead of $\sqrt{r}$ (you are welcome to consider other functions as well). How would the trade-offs be different compared to the original $\sqrt{r}$ perf function? In these cases, Would the system still benefit from heterogeneity?  Why or why not?

\begin{answer}
% Enter your answer to 1A here.
$r > \sqrt{r} > \log r $
when r is bigger, resource per one processor is bigger and tasks are better with coarse grained algorithms. \newline
when r is lower, there are too many small cores in system. \newline
Plots already show that heterogeneity is beneficial. \newline
If we increase the perf(r), the system will be more beneficial from heterogeneityi more speedup observed and vice versa.
More performance with smaller cores yield power issues, more performance with bigger cores would be beneficial. \newline
\end{answer}

\end{choice}
\newpage 

\section*{Problem 2: Domain-Specific Languages: Liszt}
Recall that Liszt was a domain-specific language that was designed for solving PDEs on meshes. Because Liszt programmers do not specify the structure of the mesh itself, the Liszt programming system has considerable flexibility in choosing the best implementation for a particular target architecture. 

In class, we discussed two very different parallel orchestration strategies that Liszt uses: (i) graph partitioning (with ghost cells), and (ii) graph coloring. For each of these approaches, please discuss the following: (i) what type of parallel target architecture is this approach well-suited for, and (ii) why is that the case.
\begin{choice}
\item What type of parallel target architecture is graph partitioning well-suited for, and why is that the case?

\begin{answer}
% Enter your answer to 2A here
Partioning is operating on its own part of the mesh. MPI system would benefit a system like this, no need for shared address space.

\end{answer}
\item What type of parallel target architecture is graph coloring well-suited for, and why is that the case?
\begin{answer}
% Enter your answer to 2B here
Graph coloring would benefit from shared address space.
\end{answer}
\end{choice}

\end{document}
