%% You can use this file to create your answer for Exercise 1.  
%% Fill in the places labeled by comments.
%% Generate a PDF document by with the command `pdflatex ex1'.

\documentclass[11pt]{article}


\usepackage{times}
\usepackage{listings}
\usepackage{enumerate}
\usepackage{courier}
\usepackage{hyperref}
\usepackage{xcolor}
\usepackage{graphicx}


%% Values that are specific to a particular term
\newcommand{\thisterm}{Spring 2021}

\newcommand{\dateassigned}{Fri.,~Apr.~2}

%% Printed form of home page that students should use
\newcommand{\visiblecoursehome}{http://www.cs.cmu.edu/\textasciitilde{}418}

%% Link to home page that will stay valid
\newcommand{\actualcoursehome}{http://www.cs.cmu.edu/afs/cs.cmu.edu/academic/class/15418-s21/www}

\newcommand{\datedue}{Fri.,~Apr.~9}





%% Page layout
\oddsidemargin 0pt
\evensidemargin 0pt
\textheight 600pt
\textwidth 469pt
\setlength{\parindent}{0em}
\setlength{\parskip}{1ex}

%% Colored hyperlink 
\newcommand{\cref}[2]{\href{#1}{\color{blue}#2}}

%% Customization to listing
\lstset{basicstyle=\ttfamily\small,language=C,morekeywords={cilk_synch,cilk_spawn}}

%% Enumerate environment with alphabetic labels
\newenvironment{choice}{\begin{enumerate}[A.]}{\end{enumerate}}
%% Environment for supplying answers to problem
\newenvironment{answer}{\begin{minipage}[c][1.5in]{\textwidth}}{\end{minipage}}


\begin{document}
                          
\vspace*{0.3in}                            
\begin{center}
\LARGE
15-418/618 \thisterm{} \\
Exercise 7
\end{center}

\begin{center}
\Large        
\begin{tabular}{ll}
\hline             
Assigned: & \dateassigned{}  \\
Due: &  \datedue{}, 11:00~pm  \\
\hline       
\end{tabular}
\end{center} 

\section*{Overview}

This exercise is designed to help you better understand the lecture
material and be prepared for the style of questions you will get on
the exams.  The questions are designed to have simple answers.  Any
explanation you provide can be brief---at most 3 sentences.  You
should work on this on your own, since that's how things will be when
you take an exam.

You will submit an electronic version of this assignment to Canvas 
as a PDF file.  For those of you familiar with the \LaTeX{} text 
formatter, you can download the template and configuation files at: 
\begin{center} 
  \cref{\actualcoursehome/exercises/ex7.tex}{\visiblecoursehome/exercises/ex7.tex}\\
  \cref{\actualcoursehome/exercises/config-ex7.tex}{\visiblecoursehome/exercises/config-ex7.tex}
\end{center} 
Instructions for how to use this template are included as comments in 
the file.  Otherwise, you can use this PDF document as your starting 
point.  You can either: 1) electronically modify the PDF, or 2) print 
it out, write your answers by hand, and scan it.  In any case, we 
expect your solution to follow the formatting of this document. 

\newpage 

\section*{Problem 1: Lock Implementations}

For the following questions, we will compare different aspects of three
lock implementations:
test-and-test-and-set, ticket-based, and array-based lock.
\begin{choice}
\item Releasing a lock requires a write.  Order the three locks based on their
interconnection traffic caused by the release.  Briefly justify your ordering.

\begin{answer}
% Enter your answer to 1A here
Test-and-test-and-set $>$ ticket-based $>$ array-based.
Test and test and set requires a write and read per processor.
$O(P) * O(P) = O(P^2)$ interconnection traffic.
Ticket-based requires one release broadcast per processor.
When other processors observe broadcast they increment counter and check if it is their serve.
O(P) interconnection traffic.
Array-based doesn't need to broadcast release to all processors. 
Every processor check its own cache line.
O(1) interconnection traffic.
\end{answer}

\item Compare the space requirements for the three lock implementations.

\begin{answer}
% Enter your answer to 1B here
Test and test and set requires one int(lock variable) for all processor.
Ticket-based requires one int(ticket variable) for all processor.
Array-based requires  one int per processor. 
Array-based $>$ ticket-based = test and test and set.
\end{answer}
\item What is the advantage provided by the ticket-based and array-based locks versus a 
test-and-test-and-set (with or without exponential backoff)?

\begin{answer}
% Enter your answer to 1C here
Ticket based gives advantage of reduced traffic. Only release needs to be broadcasted.
Array based gives further reduced traffic. No need to broadcast at all but increases space requirements.
\end{answer}
\end{choice}
\newpage 

\section*{Problem 2: Transactional Memory}

The following code sequence attempts to use transactional memory to update
a location in memory, and falls back on a lock when conflicts are detected; however,
it has a flaw where updates can be lost.  Please
explain the flaw and how the code can be updated to prevent it.  Recall that \verb!*loc += val!
is three instructions in x86 assembly.
\begin{verbatim}
void atomic_add(int* loc, int val)
{
  int result = xbegin();
  if (result == SUCCESS)
  {
    *loc += val;
    xend();
  }
  else
  {
    lock();
    *loc += val;
    unlock();
  }
}
\end{verbatim}

\begin{answer}
% Enter your asnwer to 2 here
I have thought two answers and not sure with both of them. \newline
First of them we may need to insert stop the world flags both in transactional and fallback region to ensure 
abort function behaves correctly. But when I read specifications of xend function, it says that xend rollbacks all 
values in case of terminated process, so I dont know if we have to use xabort function. \newline
Second solution is to use while(true) loop around implementation to ensure that transaction happens for sure.
\end{answer}

\end{document}
