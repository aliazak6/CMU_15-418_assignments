%% You can use this file to create your answer for Exercise 1.  
%% Fill in the places labeled by comments.
%% Generate a PDF document by with the command `pdflatex ex1'.

\documentclass[11pt]{article}


\usepackage{times}
\usepackage{listings}
\usepackage{enumerate}
\usepackage{courier}
\usepackage{hyperref}
\usepackage{xcolor}
\usepackage{graphicx}


\input{config-ex5}

%% Page layout
\oddsidemargin 0pt
\evensidemargin 0pt
\textheight 600pt
\textwidth 469pt
\setlength{\parindent}{0em}
\setlength{\parskip}{1ex}

%% Colored hyperlink 
\newcommand{\cref}[2]{\href{#1}{\color{blue}#2}}

%% Customization to listing
\lstset{basicstyle=\ttfamily\small,language=C,morekeywords={cilk_synch,cilk_spawn}}

%% Enumerate environment with alphabetic labels
\newenvironment{choice}{\begin{enumerate}[A.]}{\end{enumerate}}
%% Environment for supplying answers to problem
\newenvironment{answer}{\begin{minipage}[c][1.5in]{\textwidth}}{\end{minipage}}


\begin{document}
                          
\vspace*{0.3in}                            
\begin{center}
\LARGE
15-418/618 \thisterm{} \\
Exercise 5
\end{center}

\begin{center}
\Large        
\begin{tabular}{ll}
\hline             
Assigned: & \dateassigned{}  \\
Due: &  \datedue{}, 11:00~pm  \\
\hline       
\end{tabular}
\end{center} 

\section*{Overview}

This exercise is designed to help you better understand the lecture
material and be prepared for the style of questions you will get on
the exams.  The questions are designed to have simple answers.  Any
explanation you provide can be brief---at most 3 sentences.  You
should work on this on your own, since that's how things will be when
you take an exam.

You will submit an electronic version of this assignment to Canvas 
as a PDF file.  For those of you familiar with the \LaTeX{} text 
formatter, you can download the template and configuation files at: 
\begin{center} 
  \cref{\actualcoursehome/exercises/ex5.tex}{\visiblecoursehome/exercises/ex5.tex}\\
  \cref{\actualcoursehome/exercises/config-ex5.tex}{\visiblecoursehome/exercises/config-ex5.tex}
\end{center} 
Instructions for how to use this template are included as comments in 
the file.  Otherwise, you can use this PDF document as your starting 
point.  You can either: 1) electronically modify the PDF, or 2) print 
it out, write your answers by hand, and scan it.  In any case, we 
expect your solution to follow the formatting of this document. 

\newpage 
\section*{Problem 1: Memory Consistency}
Assume the following program segments are executed on three processors of a
multiprocessor machine. Initially before execution, all variables are equal to 0.
\begin{center}
\begin{tabular}{llcllcll}
\multicolumn{2}{c}{P1} & \makebox[0.5in]{} &
\multicolumn{2}{c}{P2} & \makebox[0.5in]{} &
\multicolumn{2}{c}{P3} \\
\texttt{E1a:} & \texttt{A = 1} && \texttt{E2a:} & \texttt{u = A} && \texttt{E3a:} & \texttt{v = B}\\
& &&            \texttt{E2b:} & \texttt{B = 1} && \texttt{E3b:} & \texttt{w = A}\\
\end{tabular}
\end{center}

\begin{choice}
\item There are totally 8 possible final states of \texttt{u}, \texttt{v}, \texttt{w} as listed.
For each one, indicate whether it is valid or invalid under a sequential consistency model.
\begin{center}
\renewcommand{\arraystretch}{1.2}
\begin{tabular}{|c|ccc|c|}
\hline
Case & \multicolumn{3}{|c|}{Final states} & Valid? (Y/N) \\
\hline
Case 1: & \texttt{u} = 0 & \texttt{v} = 0 & \texttt{w} = 0 & 
%% Indicate whether valid (Y/N)
Yes
\\
\hline
Case 2: & \texttt{u} = 0 & \texttt{v} = 0 & \texttt{w} = 1 &  
%% Indicate whether valid (Y/N)
Yes
\\
\hline
Case 3: & \texttt{u} = 0 & \texttt{v} = 1 & \texttt{w} = 0 &  
%% Indicate whether valid (Y/N)
Yes
\\
\hline
Case 4: & \texttt{u} = 0 & \texttt{v} = 1 & \texttt{w} = 1 & 
%% Indicate whether valid (Y/N)
Yes
\\
\hline
Case 5: & \texttt{u} = 1 & \texttt{v} = 0 & \texttt{w} = 0 &  
%% Indicate whether valid (Y/N)
Yes
\\ 
\hline
Case 6: & \texttt{u} = 1 & \texttt{v} = 0 & \texttt{w} = 1 &  
%% Indicate whether valid (Y/N)
Yes
\\
\hline
Case 7: & \texttt{u} = 1 & \texttt{v} = 1 & \texttt{w} = 0 &  
%% Indicate whether valid (Y/N)
No
\\
\hline
Case 8: & \texttt{u} = 1 & \texttt{v} = 1 & \texttt{w} = 1 &  
%% Indicate whether valid (Y/N)
Yes
\\
\hline
\end{tabular}
\end{center}

\item 
Choose one of the final states that you think is invalid under sequential consistency.
Prove that it is invalid.  (Refer to the operations by the event labels 
\texttt{E1a}, \texttt{E2b}, etc.)

\begin{answer}
%% Provide your answer to 1B here
Case 7 is invalid.  Consider the following execution:
\begin{itemize}
\item P1 executes and A is set to 1.
\item P2 executes and u is set to 1(valid). P3 executes and v is set to 0 (unvalid).
\item P3 executes and v is set to 1, w is set to 1.


\end{itemize}
\end{answer}

\item To maintain the same possible final states (as in sequential
  consistency) under weaker memory consistency model, one way is to
  add fences that guarantee that all reads/writes prior to the fence
  complete before any read/write after the fence. There are 8 possible
  placements to add fences:
\begin{center}
\begin{tabular}{llcllcll}
\multicolumn{2}{c}{P1} & \makebox[0.5in]{} &
\multicolumn{2}{c}{P2} & \makebox[0.5in]{} &
\multicolumn{2}{c}{P3} \\
\multicolumn{2}{c}{\textit{${\it Fence}_1$}} &&  \multicolumn{2}{c}{\textit{${\it Fence}_2$}} && \multicolumn{2}{c}{\textit{${\it Fence}_3$}} \\
\texttt{E1a:} & \texttt{A = 1} && \texttt{E2a:} & \texttt{u = A} && \texttt{E3a:} & \texttt{v = B}\\
\multicolumn{2}{c}{\textit{${\it Fence}_4$}} &&  \multicolumn{2}{c}{\textit{${\it Fence}_5$}} && \multicolumn{2}{c}{\textit{${\it Fence}_6$}} \\
& &&            \texttt{E2b:} & \texttt{B = 1} && \texttt{E3b:} & \texttt{w = A}\\
\multicolumn{2}{c}{} &&  \multicolumn{2}{c}{\textit{${\it Fence}_7$}} && \multicolumn{2}{c}{\textit{${\it Fence}_8$}} \\
\end{tabular}
\end{center}
 To minimize the number of necessary fences
  (while maintaining sequential consistency), where would you place
  the fences?  List a minimal set of fences, and argue that removing any one of
  these could lead to a consistency violation.

\begin{answer}
%% Provide your answer to 1C here
Fence8 should be enough to maintain sequential consistency. Only Case7 should not exist.
If we fence after E3b, then E3a cannot execute without E3b. So consistency is maintained.

\end{answer}

\item
Suppose you could use also use a storage fence or a load fence.
Could you replace any of the full memory fences in your list with one
of these?

\begin{answer}
%% Provide your answer to 1D here
I have one fence in my list. I can replace it with a storage fence for w, and E3b, E3a will be executed in order.
\end{answer}

\end{choice}
\newpage

\section*{Problem 2: Cache Coherency}
You are hired by Intel to design an invalidation based cache-coherent processor with a large number of cores.
The 
  main application for this processor will be to train a chess AI by playing games against itself for experiments. It is critical that it
  doesn't play too many games during training to maintain the validity of the experiment, so we store how many
  games are played in a shared counter:
  \begin{lstlisting}
  int num_games = 100000;
  int games_played = 0;

  // This code is run on each core
  while (true) {
    // Atomically get value of count prior to increment, and
    // write incremented value to games_played
    int val = atomic_add(&games_played, 1);   

    if (val > num_games) break;
    play_game()
  }
  \end{lstlisting}

  Unsure on how to design the cache coherence of this processor, you check with one of the senior designers. She suggests that you 
  should use a bus-based, snooping coherence implementation, rather than a directory-based protocol, because 
 the broadcasting it performs would be more
  efficient in this case. Do you agree or disagree? Why or why not?

\begin{answer}
% Enter your answer to 2 here
I agree with senior designer because all cores need to know the value of games played. 
So a bus-based protocol is more efficient because of this equal workload distribution.
\end{answer}
\end{document}

